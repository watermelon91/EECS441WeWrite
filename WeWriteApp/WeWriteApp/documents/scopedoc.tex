\documentclass[10pt]{article}
\usepackage{fancyvrb}
\usepackage{listings}
\usepackage{mathtools}

\textwidth=400pt
\voffset = 0in
\marginparwidth = 100pt
\topmargin = 0pt
\oddsidemargin = 32pt
\textheight = 648pt
\headsep = 10pt
\marginparsep = 10pt

\CustomVerbatimEnvironment{myverb}{Verbatim}
    {frame=leftline,framesep=8pt,framerule=14pt,rulecolor=\color{lightgray},numbers=left}

\title{EECS441: Team \#5}
\author{ University of Michigan \\ Ann Arbor, MI}
\date{}

\begin{document}
\maketitle

\begin{tabular}{ll}
	\textbf{To:} & Rod Johnson, Instructor, TchclComm 497, University of Michigan \\
	\textbf{From:} & Kevin Tan, Team Member \\
			     &  Yijia Tang, Team Member \\
         \textbf{Subject:} & WeWriteApp Scope Document \\
	\textbf{Date:} & September 26, 2013 \\
    \textbf{CC:} & Elliot Soloway, Instructor, EECS441, University of Michigan \\
\end{tabular}

\section*{?}
version box here.

\section{Overview}
The first project in the Mobile Apps Development class (EECS441, taught by Elliot Soloway), 
is to design a collaborative text editor for mobile devices which allows multiple users to
 edit a single document at the same time. In this document, we will lay out the strategies
 and basic approaches we took to complete the project.

\newpage

\section{Discussion}
\paragraph{Introduction} ~ \\
\noindent
The goal for this project is to build a mobile application based on the Collabrify API 
(Application Programming Interface) that will be used in the K-12 classroom. This will
 encourage class participation and also have students with working and learning with
 technology from a young age.

\paragraph{} ~\\
\noindent
The motivation for our contributions to this project is to familiarize us with using an API
 or framework that we may not necessarily be familiar with already, to simulate the work-place, 
where we may have to use corporate (proprietary) libraries that we have not seen before.  
In addition, this is a small-scale project that serves to warm us up into the mobile-apps world.
\end{document}
